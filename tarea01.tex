\documentclass[12pt]{article}
\usepackage{graphicx}
\usepackage{caption}
\usepackage{subcaption}
\usepackage{tikz}
\usepackage{venndiagram}
\usepackage{venndiagram}
\usepackage{tcolorbox}
\usepackage{listings}
\usepackage{enumitem}
\usepackage{amsmath}
\usepackage{amssymb}
\usepackage{colortbl}
\usepackage{xcolor}
\usepackage[margin=1cm, top=1.5cm, bottom=1.5cm]{geometry}

\tcbuselibrary{breakable}

\title{\textbf{Gráficas y Juegos: Tarea 02}}
\author{Martínez Méndez Ángel Antonio\\Valencia Morales Indra Gabriel\\Rendón Ávila Jesús Mateo}
\date{\today}

\begin{document}

\maketitle
\begin{center}
\vspace{3cm}
\includegraphics[width=0.195\textwidth]{Escudo.png}
\hspace{0.5cm}
\includegraphics[width=0.2\textwidth]{logo_ciencias.png}
\end{center}
\begin{center}
    \vspace{1cm}
    Universidad Nacional Autónoma de México\\
    Facultad de Ciencias\\
    Profesor: César Hernández Cruz\\
\end{center}

\newpage

%
% Ejercicio 1
%
\textbf{1.} Sea $(R, +, \ast)$ un anillo. Demostrar mediante definiciones que $R$ es conmutativo si y sólo si
$\forall x, y, \in R$ se cumple: $(x + y) \ast (x - y) = x^2 - y^2$\\

$\Longrightarrow)$ $Hipotesis$. R es conmutativo.\\

$P.D$. Bajo la operaicón de la operación del producto se cumple que $\forall x, y \in R$ $(x + y) \ast (x - y) = x^2 - y^2$\\

Sea $x, y, \in R$, entonces\\

\begin{align*}
    (x + y) \ast (x - y) &= x \ast (x - y) + y \ast (x - y)\\
    &= xx -xy +yx -yy \textit{ (por conmutatividad)}\\
    &= xx -xy +xy -yy\\
    &= xx - yy\\
    &= x^2 -y^2\\
\end{align*}

$\Longleftarrow)$ $Hipotesis$:  $\forall x, y \in R$ se cumple que $(x + y) \ast (x - y) = x^2 - y^2$.\\

$P.D$. $R$ es conmutativo. $i.e$ $\forall x, y \in R$ se cumple $x \ast y = y \ast x$\\

Sea $x, y \in R$\\

Por hipotesis
\begin{align*}
                (x + y) \ast (x - y) &= x^2 - y^2\\
                x \ast (x -y) + y \ast (x -y) &= x^2 - y^2\\
                x^2 -xy + yx -y^2 &= x^2 - y^2 \textit{ (por cancelación)}\\
                -xy + yx &= 0\\
                yx &= xy\\
              \end{align*}
\vspace{1cm}

%
% Ejercicio 2
%
\textbf{2.} Considera la relación $\sim$ usada para definir a $\mathbb{Z}$ y $k \in \mathbb{N}$. Demuestra que:
    \begin{enumerate}[label=\alph*)]
        \item $\overline{(k, 0)} = \{(a, b) \in \mathbb{N} \times \mathbb{N} \mid \text{ existe } n \in \mathbb{N} \text{ tal que } a = k + n \text{ y } b = n\}$.

        \item Usando el inciso previo, escribe por extensión el conjunto

    \end{enumerate}
\vspace{1cm}

%
% Ejercicio 3
%
\textbf{3.} Muestra los siguientes incisos referentes a orden en $\mathbb{Z}$.
    \begin{enumerate}[label=\alph*)]
        \item Sean $a, b \in \mathbb{Z}^+$ tales que $a \leq b$. Usando definiciones, prueba que si $0 < n$, enotnces $a^n \leq b^n$.

        \item Si $ a \leq 0$ y $0 < b$, entonces $ab \leq a$

        \item Si $a \leq b$ y $c < d$, muestra con definiciones que $a - d < b - c$.

    \end{enumerate}
\vspace{1cm}

%
% Ejercicio 4
%
\textbf{4.} Calcula el cociente y el residuo de los siguientes incisos.

    \begin{enumerate}[label=\alph*)]

        \item 175 entre 46.

        \item 20145 entre 1050.

        \item -326 entre 40.

    \end{enumerate}
\vspace{1cm}

%
% Ejercicio 5
%
\textbf{5.} Muestra los siguientes incisos referente a divisibilidad en $\mathbb{Z}$

    \begin{enumerate}[label=\alph*)]
        \item Sean $a$ y $b$ dos enteros. Muestra que $|a| | b$ si y sólo si $a | b$ y $-a |b$.

        \item Muestra usando definiciones que si $a | b$ y $a | b + c$, entonces $a | c$.

        \item Muestra usando definiciones que si $a , b \in \mathbb{Z}$ y $0 \leq n$, entonces $a - b | a^n - b^n$.
           \end{enumerate}

\vspace{1cm}

%
% Ejercicio 6
%
\textbf{6.} Muestra mediante inducción matemática lo siguiente. Si $a | b_1, a | b_2, \dots, a | b_n$, entonces $a | b_1 + \dots + a | b_n$.\\

Mostraremos por induccion sobre $n$ que si $a | b_1, a | b_2, \dots, a | b_n$, entonces $a | b_1 + \dots + a | b_n$.

Base inductiva. Sea $n = 2$ si $a | b_1, a | b_2$ entonces $a | b_1 + a | b_2$

Por definicion de divisor, como $a | b_1$ entonces existe un único $r \in Z$ tal que $ar = b_1$
igualmente como $a | b_2$ entonces existe un único $s \in Z$ tal que $as = b_2$
es decir $a(r+s) = b_1 + b_2$
por lema 2.1.1 Si $a | b$ y $a | c$, entonces $a | b + c$. Podemos afirmar que $a | b_1 + b_2$

Hipotesis inductiva. supondremos que $k$ es un natural tal que $k \geq 2$ si $a | b_1, a | b_2, \dots, a | b_k$ entonces $a | b_1 + \dots + b_k$ es decir, existe $at = b_1 + \dots + b_k$.

Paso inductivo. Sea $k + 1$ un natural tal que $k + 1 \geq 2$ si $a | b_1, a | b_2, \dots, a | b_k, a | b_k+1$ entonces $a | b_1 + \dots + b_k + b_k+1$

Como por hipotesis inductiva $k \geq 2$, entonces podemos afirmar que $k+1 \geq 3$
Por definicion de divisor, como $a | b_k+1$ entonces existe un único $v \in Z$ tal que $av = b_k+1$
por hipotesis inductiva sabemos que $a | b_1, a | b_2, \dots, a | b_k$ entonces $a | b_1 + \dots + b_k$, es decir, existe $at = b_1 + \dots + b_k$
integrando $b_k+1$ entonces $a | b_1, a | b_2, \dots, a | b_k, a | b_k+1$ entonces $a | b_1 + \dots + b_k + b_k+1$, es decir, $at + av = b_1 + \dots + b_k + b_k+1$.
reescribiendo tenemos $a(t + v) = b_1 + \dots + b_k + b_k+1$.
Concluyendo por principio de induccion matematica  para todo $k \geq 2$ si $a | b_1, a | b_2, \dots, a | b_k$, entonces $a | b_1 + \dots + b_k $.
\vspace{1cm}

Usando lo anterior, muestra que si $a | b_1, a | b_2, \dots, a | b_n$, entonces para toda $c_1, \dots, c_n \in \mathbb{Z}$ se cumple que $a | c_1 b_1 + \dots + c_n b_n$
\vspace{1cm}
Sea $a | b_1, a | b_2, \dots, a | b_n$ entonces existe $ar_1 = b_1 , ar_2 = b_2 ... ar_n = b_n$.
por el lema 2.1.2 Si $a | b$, entonces $ac | bc$. Luego $ac_1 | c_1 b_1, ac_2 | b_2, . . . , ac_n | c_n b_n$ por definicion de divisiblidad $a c_1 r_1 = c_1 b_1 , a c_2 r_2 =  c_2 b_2 ... a c_n r_n = c_n b_n$.
Como anteriormente vimos $a | b_1 + \dots + b_k$, es decir, $at = b_1 + \dots + b_k$ aplicando el lema 2.1.2 nuevamente podemos decir $avt = c_1 b_1 + c_2 b_2 ... c_n b_n$
aplicando una vez mas definicion de divisibilidad
se cumple que $a | c_1 b_1 + \dots + c_n b_n$

%
% ejercicio 7
%
\textbf{7.} Sean $a$ y $b$ dos enteros. Muestra que si $13 | 5a + 8b$, entonces $13|31a - 5b$\\

\vspace{1cm}

%
% Ejercicio 8
%
\textbf{8.} Sean $a$ y $b$ dos enteros no nulos y $d \in \mathbb{Z}^+$ tal que $d | a$ y $d | b$. Muestra que $\frac{ab}{d} = \frac{ba}{d}$.

\vspace{1cm}

%
% Ejercicio 9
%
\textbf{9.} Calcula los siguientes incisos.
\begin{enumerate}[label=\alph*)]
    \item Calcula 723 en base 7.
        \begin{align*}
            723 &= 103 \ast 7 + 2\\
            103& = 14 \ast 7 + 5\\
            14 &= 2 \ast 7 + 0\\
            2 &= 0 \ast 7 + 2\\
        \end{align*}

        Con ello tenemos que $723$ en base 7 es: $(2052)_7$.

    \item Calcula 27 en base 2.
        \begin{align*}
            27 &= 13 \ast 2 + 1\\
            13 &= 6 \ast 2 + 1\\
            6 &= 3 \ast 2 + 0\\
            3 &= 1 \ast 2 + 1\\
            1 &= 0 \ast 2 + 1\\
        \end{align*}

        Con ello tenemos que $27$ en base 2 es: $(11011)_2$.\\

    \item Calcula $(1076)_8 + (2076)_8$.\\

        Primero vamos a pasar a base 10 ambas cantidades.\\

        $(1076)_8$ a base 10.\\
        $8^0 \ast 6 + 8^1 \ast 7 + 8^2 \ast 0 + 8^3 \ast 1 = 6 + 56 + 0 + 512 = 574$\\

        $(2076)_8$ a base 10.\\
        $8^0 \ast 6 + 8^1 \ast 7 + 8^2 \ast 0 + 8^3 \ast 2 = 6 + 56 + 0 + 1024 = 1086$\\

        $574 + 1086 = 1660$\\

        Podemos entonces pasar 1660 a base 8\\
        \begin{align*}
            1660 &= 207 \ast 8 + 4\\
            207 &= 25 \ast 8 + 7\\
            25 &= 3 \ast 8 + 1\\
            3 &= 0 \ast 8 + 3\\
        \end{align*}

        Podemos entonces decir que la suma de $(1076)_8 + (2076)_8 = (3174)_8$.
\end{enumerate}

\vspace{1cm}

%
% Ejercicio 10
%
\textbf{10.} Un profesor de matemáticas califica los exámenes de la siguiente manera: el primer problema vale un
punto, el segundo 2, el tercero 4, el cuarto 8 y así sucesivamente. Un problema, o está bien o está mal, no hay
término medio. Un alumno aprueba si al menos la mitad de todos los problemas están bien. Un estudiante
obtuvo en el examen de junio, que constaba de 10 problemas, 581 puntos. Determina qué problemas hizo bien
y si aprobó el examen o no.\\

Primero, tengamos el listado de puntajes a mano: $1, 2, 4, 8, 16, 32, 64, 128, 256, 512$.\\

Podemos ver que estamos en base 2, por lo que si trasladamos 581 a base 2 podriamos ver los problemas correctos.
\begin{align*}
    581 &= 290 \ast 2 + 1\\
    290 &= 145 \ast 2 + 0\\
    145 &= 72 \ast 2 + 1\\
    72 &= 36 \ast 2 + 0\\
    36 &= 18 \ast 2 + 0\\
    18 &= 9 \ast 2 + 0\\
    9 &= 4 \ast 2 + 1\\
    4 &= 2 \ast 2 + 0\\
    2 &= 1 \ast 2 + 0\\
    1 &= 0 \ast 2 + 1\\
\end{align*}

De lo anterior podemos ver que 581 es igual a $(1001000101)_2$. Así vemos que se obtuvieron bien el problema 10, 7, 3 y 1, que son
$512 + 64 + 4 + 1 = 581$. Por lo que, al tener 4 problemas correctos, el alumno no aprobó el examen.

%
% Extras. Ejercicio 2
%
\textbf{2.} Sean $(R, +, ·)$ un anillo conmutativo, $0$ el neutro aditivo de $R$ y $1$ el neutro multiplicativo de $R$. Muestra
que $R = {0}$ si y solo si $1 = 0$. \\

Demostraremos que $R = {0}$ si y solo si $1 = 0$.
Procederemos por doble implicacion para demostrar nuestro bicondicional

<--
Supongamos que R = {0}
Como $R$ es un anillo conmutativo por hipotesis tenemos que $1$ es el neutro multiplicativo de $R$, por lo que $1 \in R$ como $R = {0}$ entonces podemos concluir
$1 = 0$

-->
Supongamos que $1 = 0$
Notemos que los unicos objetos que podemos asegurar que pertenecen a $1 \in R$ y $0 \in R$, procederemos por contradiccion es decir que existe algun $c \neq 1$ y $c \neq 0$ entonces
por existe $1 \in R$ tal que para todo $c \in R$, $1 · c = c$ y $c · 1 = c$.
notemos que como $1 = 0$ entonces $0$ tambien es neutro multiplicativo por lo que substituyendo
$0 · c = 0$ y $c · 0 = 0$ !
La contradiccion anterior nos dice que al existir un $c \neq 1$ y $c \neq 0$ no se puede cumplir la propiedad de neutro multiplicativo
De lo anterior podemos concluir que el unico miembro perteniente a $R$ es $0$, concluyendo $R = {0}$

\end{document}
