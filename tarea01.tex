\documentclass{article}
\usepackage{polyglossia}
\usepackage{enumerate} 
\usepackage{fixltx2e}
\usepackage{tabularx}
\usepackage{geometry}
\usepackage{xfp}
\usepackage{pgf}
\usepackage{siunitx}
\usepackage{xcolor}
\usepackage{fancyhdr}
\usepackage{titlesec}
\usepackage{listings}
\usepackage{enumitem}
\usepackage{amsmath}
\usepackage{amssymb}
\usepackage{setspace}
\newcommand{\contradiction}{\mathrel{\text{\lightning}}}
\setlength{\parindent}{0cm}

% Color personalizado para el fondo
\definecolor{bgcolor}{rgb}{0.95,0.95,0.92}
\definecolor{mycolor}{rgb}{0.1,0.2,0.7}

\geometry{top=4cm, head=4cm, left=2.5cm, right=2.5cm,a4paper}
\pagestyle{fancy}
\fancyhead[R]{\textcolor{gray}{Universidad Nacional Autónoma de México\\ Facultad de Ciencias\\
Licenciatura en Ciencias de la Computación\\
Álgebra Superior II\\}}


\setdefaultlanguage{spanish}
\title{\textbf{Tarea 01\\ Enteros}
\vspace{1cm}}
\author{Larios Ponce Héctor Manuel\\Valencia Morales Indra Gabriel}
\date{\textit{\today}}

\titleformat{\section}
{\normalfont\Large\bfseries}
{\textcolor{mycolor}{\thesection}}
{1em}
{\textcolor{mycolor}}
\begin{document}
\maketitle

\pagebreak


\section{Sea $k$ un número natural. Demuestra por inducción\\matemática que si $n \geq 2$, entonces $\sum\limits_{i=1}^n k = nk$}

\begin{minipage}{\textwidth}
\setlength{\leftskip}{0.5cm}
Demostraremos por inducción matemática sobre $n$, con $n \in \mathbb{N}$, que si $n\geq 2$, entonces $\sum\limits{i=1}^n k = nk$

\vspace{0.5cm}

\textbf{Base de inducción:}

\setlength{\leftskip}{1cm}
Si $n  = 2$ entonces:

Calculando la suma tenemos: $\sum\limits_{i=1}^2 k = k + k = 2k$\\
$\therefore$ La base de inducción se satisface.

\vspace{0.5cm}
\setlength{\leftskip}{0.5cm}
\textbf{Hipótesis de inducción:}

\setlength{\leftskip}{1cm}
Supongamos que para algún $n \in \mathbb{N}$, que si $n \geq 2$ entonces $\sum\limits_{i = 1}^n k = nk$

\vspace{0.5cm}
\setlength{\leftskip}{0.5cm}
\textbf{Paso inductivo:}

\setlength{\leftskip}{1cm}
Si $n \in \mathbb{N}$, entonces $\sum\limits_{i=1}^{n+1} k = (n + 1)k$

\vspace{0.5cm}
\textit{Dem.}

\vspace{0.5cm}
Sea $n + 1 \in \mathbb{N}$, podemos expresar $(n + 1)k$ como $\sum\limits_{i=1}^n k + k$, entonces tenemos:

$\sum\limits_{i=1}^{n+1} k = \sum\limits_{i=1}^n k + k$

\vspace{0.5cm}
$\sum\limits_{i=1}^{n+1} k = nk + k$ \textit{-Por hipótesis de inducción, $nk = \sum\limits_{i=1}^n k$}

\vspace{0.5cm}
$\sum\limits_{i=1}^{n+1} k = (n + 1)k$ \textit{-Factorizando k}

\vspace{0.5cm}
$\therefore$ Por el principio de inducción matematica queda demostrado que $\forall n$ si $n\geq 2$, entonces $\sum\limits_{i=1}^n k$
\end{minipage}
\end{document}